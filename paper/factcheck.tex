\documentclass{chi2009}
\usepackage{times}
\usepackage{url}
\usepackage{graphics}
\usepackage{color}
\usepackage[pdftex]{hyperref}
\hypersetup{%
pdftitle={Your Title},
pdfauthor={Your Authors},
pdfkeywords={your keywords},
bookmarksnumbered,
pdfstartview={FitH},
colorlinks,
citecolor=black,
filecolor=black,
linkcolor=black,
urlcolor=black,
breaklinks=true,
}
\newcommand{\comment}[1]{}
\definecolor{Orange}{rgb}{1,0.5,0}
\newcommand{\todo}[1]{\textsf{\textbf{\textcolor{Orange}{[[#1]]}}}}

\pagenumbering{arabic}  % Arabic page numbers for submission.  Remove this line to eliminate page numbers for the camera ready copy

\begin{document}
% to make various LaTeX processors do the right thing with page size
\special{papersize=8.5in,11in}
\setlength{\paperheight}{11in}
\setlength{\paperwidth}{8.5in}
\setlength{\pdfpageheight}{\paperheight}
\setlength{\pdfpagewidth}{\paperwidth}

% use this command to override the default ACM copyright statement 
% (e.g. for preprints). Remove for camera ready copy.
\toappear{Submitted for review to CHI 2009.}

\title{Natural Language Fact Checker}
\numberofauthors{2}
\author{
  \alignauthor Joshua Kraunelis\\
    \affaddr{University of Massachusetts Lowell}\\
    \affaddr{1 University Ave, Lowell, MA 01852}\\
    \email{jkraunel@cs.uml.edu}
  \alignauthor Wesley Nuzzo\\
    \affaddr{University of Massachusetts Lowell}\\
    \affaddr{1 University Ave, Lowell, MA 01852}\\
    \email{uraniumcronorum@gmail.com}
}

\maketitle

\begin{abstract}
  In this paper we describe the formatting requirements for SIGCHI
  Conference Proceedings, and offer recommendations on writing for the
  worldwide SIGCHI readership.  Please review this document even if
  you have submitted to SIGCHI conferences before, for some format
  details have changed relative to previous years. These include the
  formatting of table captions, the formatting of references, and a
  requirement to include ACM DL indexing information.
\end{abstract}

\keywords{automated fact checking} 

\section{Introduction}

Fact checking is the task of finding evidence to validate or invalidate a given claim.   We propose to create a system which can verify basic types of claims automatically, without the need for a human fact checker.

Fullfact.org defines three main approaches to automated fact checking: reference, machine learning, and context.  The reference approach relies on some official source of information to validate the claim.  The machine learning approach uses some model of how the world works to estimate the likelihood of a claim.  The contextual approach looks at the prevalence of a claim to decide how likely it might be. \ref{fullfact}

Hassan et al. present a system for identifying claims automatically via machine learning.  Their software, ClaimBuster, performs natural language processing tasks such as sentiment analysis, tokenization, part of speech tagging, and entity resolution to classify a sentence as non-factual, unimportant factual, or check-worthy factual.  This system reduces the amount of time spent searching for claims to be validated, though it does not perform the validation itself.  \ref{hassan}

Ciampaglia et al. take a graph-theoretic approach to automated fact checking.  They encode knowledge from Wikipedia into graph data structures called knowledge graphs that, together, form larger structures called knowledge networks.  They show that traditional graph/network analysis techniques such as shortest path computation can be used to efficiently evaluate the truthfulness of a claim. \ref{ciampaglia}


\section{Project Description}

On each page your material (not including the page number) should fit
within a rectangle of 18 x 23.5 cm (7 x 9.25 in.), centered on a US
letter page, beginning 1.9 cm (.75 in.) from the top of the page, with
a .85 cm (.33 in.) space between two 8.4 cm (3.3 in.) columns.  On an
A4 page, use a text area of the same dimensions (18 x 23.5 cm.), again
centered.  Right margins should be justified, not ragged. Beware,
especially when using this template on a Macintosh, Word can change
these dimensions in unexpected ways.

\section{Analysis of Results}

Prepare your submissions on a word processor or typesetter.  Please
note that page layout may change slightly depending upon the printer
you have specified.  For this document, printing to Adobe Acrobat PDF
Writer was specified.  In the resulting page layout, Figure 1 appears
at the top of the left column on page 2, and Table 1 appears at the
top of the right column on page 2.  You may need to reposition the
figures if your page layout or PDF-generation software is different.

\subsection{Discussion}

Your paper's title, authors and affiliations should run across the
full width of the page in a single column 17.8 cm (7 in.) wide.  The
title should be in Helvetica 18-point bold; use Arial if Helvetica is
not available.  Authors' names should be in Times Roman 12-point bold,
and affiliations in Times Roman 12-point (note that Author and
Affiliation are defined Styles in this template file).

To position names and addresses, use a single-row table with invisible
borders, as in this document.  Alternatively, if only one address is
needed, use a centered tab stop to center all name and address text on
the page; for two addresses, use two centered tab stops, and so
on. For more than three authors, you may have to place some address
information in a footnote, or in a named section at the end of your
paper. Please use full international addresses and telephone dialing
prefixes.  Leave one 10-pt line of white space below the last line of
affiliations.

\section{Conclusions}

Every submission should begin with an abstract of about 150 words,
followed by a set of keywords. The abstract and keywords should be
placed in the left column of the first page under the left half of the
title. The abstract should be a concise statement of the problem,
approach and conclusions of the work described.  It should clearly
state the paper's contribution to the field of HCI.

The first set of keywords will be used to index the paper in the
proceedings. The second set are used to catalogue the paper in the ACM
Digital Library. The latter are entries from the ACM Classification
System~\cite{acm_categories}.  In general, it should only be necessary
to pick one or more of the H5 subcategories, see
http://www.acm.org/class/1998/H.5.html

\section{Acknowledgements}

Please use a 10-point Times Roman font or, if this is unavailable,
another proportional font with serifs, as close as possible in
appearance to Times Roman 10-point. The Press 10-point font available
to users of Script is a good substitute for Times Roman. If Times
Roman is not available, try the font named Computer Modern Roman. On a
Macintosh, use the font named Times and not Times New Roman. Please
use sans-serif or non-proportional fonts only for special purposes,
such as headings or source code text.

\bibliographystyle{abbrv}
\bibliography{factcheck}

\end{document}
